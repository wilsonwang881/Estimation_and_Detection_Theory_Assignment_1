\documentclass[11pt,letterpaper,titlepage]{article}

\usepackage{geometry}
\geometry{left=1.5cm,right=1.5cm,top=1.5cm,bottom=2.5cm}

\usepackage{setspace}
\onehalfspacing

\usepackage{fancyhdr}

\usepackage{amsmath}

\usepackage{amssymb}

\usepackage{booktabs}

\usepackage{pifont}

\pagestyle{fancy}
\lhead{}
\rhead{}
\lfoot{ECEN 662 Estimation and Detection Theory}
\cfoot{\thepage}
\rfoot{@Lei Wang}
\renewcommand{\headrulewidth}{0pt}
\renewcommand{\headwidth}{\textwidth}
\renewcommand{\footrulewidth}{0.4pt}
\newcommand{\RomanNumeralCaps}[1]
    {\MakeUppercase{\romannumeral #1}}

\begin{document}

\begin{enumerate}
    \item $$ \lim_{n\to\infty} \frac{n!}{n^{n+(1/2)}}e^{-n} = \lim_{n\to\infty} \frac{n!}{n^{n}} \cdot \frac{1}{n^{1/2}} \cdot e^{-n} = \lim_{n\to\infty} \frac{n}{n} \cdot \frac{n-1}{n} \cdot \frac{n-2}{n} \cdot \frac{n-3}{n} ... \frac{1}{n} \cdot \frac{1}{n^{1/2}} \cdot e^{-n} $$

    $$ \text{Note as } {n\to\infty}, \frac{n}{n} = 1, \frac{n-1}{n}\to 1...\frac{1}{n}\to 0, \text{ and } \frac{1}{n^{1/2}} \to 0, e^{-n} \to 0 $$
    
    $$ \text{therefore } \lim_{n\to\infty} \frac{n!}{n^{n+(1/2)}}e^{-n} = 0 $$
    
    \item \begin{enumerate}
        \item Drawing the numbers with replacement means that if there were $ n $ numbers, a total number of $ n^n $ drawing scenarios can be reached: $ n $ positions in a draw, each with $ n $ choices. Hence $ n^n $ scenarios.
        
        \begin{table}[ht]
        \centering
        \begin{tabular}{@{}lcc@{}}
        \toprule
        Scenario & Number of appearance & Probability associated with the draw average \\ \midrule
        No repeated number         &    $ n! $                   &   $ \frac{n!}{n^n} $                                           \\ \midrule
        One repeated number         &      $ \frac{n!}{2!} $                 &              $ \frac{n!}{2!n^n} $                                \\ \midrule
        m repeated number         &     $ \frac{n!}{(m+1)!} $                 &           $ \frac{n!}{(m+1)!n^n} $                                       \\ \bottomrule
        \end{tabular}
        \end{table}
        
        From the above table, the scenario that has the highest probability associated with its average is the no repeated number scenario, which has the probability of $ \frac{n!}{n^n} $ .
        
        \item
        
        \item At each position in the draw, each number has a probability of $ 
        \frac{1}{n} $ to be placed at that position. Hence, the probability that a number is missing from one particular position in the draw is $ (1-\frac{1}{n}) $. A draw of $ n $ numbers has $ n $ positions. To avoid appearing in any of the $ n $ positions, the probability is $ (1-\frac{1}{n})^n $. According to the definition of $ e $:
        
        $$ e = \lim_{n\to\infty} (1 + \frac{1}{n})^n $$
        
        $$ e^{-1} = \lim_{n\to\infty} (1 + \frac{1}{n})^{-n} = \lim_{n\to\infty} (\frac{1}{1+\frac{1}{n}})^{n} = \lim_{n\to\infty} (\frac{n}{n+1})^{n} = \lim_{n\to\infty} (\frac{n+1}{n+1} - \frac{1}{n+1})^{n} $$
        
        As $ n \to\infty, n+1 \to n, $
        
        Therefore:
        
        $$ \lim_{n\to\infty} (\frac{n+1}{n+1} - \frac{1}{n+1})^{n} = \lim_{n\to\infty} (1 - \frac{1}{n})^{n} = \lim_{n\to\infty} (1-\frac{1}{n})^n $$
        
    \end{enumerate}
    
    \newpage
    
    \item Draw a table to list the distribution, sum up the corresponding probabilities:
    
    \begin{table}[ht]
    \centering
    \begin{tabular}{@{}lllll@{}}
    \toprule
    Number of balls in each cell & Associated probability for such distribution & ${X_3=0}$ & ${X_3=1}$ & ${X_3=2}$ \\ \midrule
    $(1,1,1,1,1,1,1)$              &  $\frac{1}{13}$                           & \ding{52} &  &  \\ \midrule
    $(2,1,1,1,1,1)$                &  $\frac{1}{13}$                             & \ding{52} &  &  \\ \midrule
    $(3,1,1,1,1)$                 &   $\frac{1}{13}$                          &  & \ding{52} &  \\ \midrule
    $(4,1,1,1)$                    &  $\frac{1}{13}$                         & \ding{52} &  &  \\ \midrule
    $(5,1,1)$                     &   $\frac{1}{13}$                                & \ding{52} &  &  \\ \midrule
    $(6,1)$                        &  $\frac{1}{13}$                           &\ding{52}  &  &  \\ \midrule
    $(7)$                          &  $\frac{1}{13}$                              & \ding{52} &  &  \\ \midrule
    $(2,2,1,1,1)$                  &  $\frac{1}{13}$                           & \ding{52} &  &  \\ \midrule
    $(2,3,1,1)$                    &  $\frac{1}{13}$                                 &  &\ding{52}  &  \\ \midrule
    $(2,4,1)$                      &  $\frac{1}{13}$                                 &\ding{52}  &  &  \\ \midrule
    $(3,3,1)$                      &  $\frac{1}{13}$                              &  &  & \ding{52} \\ \midrule
    $(2,5)$                        &  $\frac{1}{13}$                           &  &  &  \\ \midrule
    $(3,4)$                        &  $\frac{1}{13}$                                 &  & \ding{52} &  \\ \bottomrule
    \end{tabular}
    \end{table}
    
    Hence:
    
    \begin{table}[ht]
    \centering
    \begin{tabular}{ll}
    \toprule
    Pr(${X_3=0}$) & $\frac{8}{13}$ \\ \midrule
    Pr(${X_3=1}$) & $\frac{3}{13}$ \\ \midrule
    Pr(${X_3=2}$) & $\frac{1}{13}$ \\ \midrule
    Pr(${X_3=n|n\geq3}$) & $0$ \\ \bottomrule
    \end{tabular}
    \end{table}
    
    \item To get the moment generating function of a random variable $ X $, we do:
    
    $$ M_{X} (s) = E[e^{sX}] = \int_{-\infty}^{\infty} e^{s x} f_{X} (x) dx $$
    
    Note for Laplace transform:
    
    $$ L\{f(x)\}  = \int_{-\infty}^{\infty} e^{-sx} f(x) dx $$
    
    Therefore:
    
    $$ M_{X} (s) = L\{f_{X} (x)\}(-s) $$
    
    $ M_{X} (s) $ can be made into the following form:
    
    $$ M_{X} (-s) = \frac{2}{3} \cdot \frac{1}{s + 2} + 2 \cdot \frac{1}{s + 3} $$
    
    Note $ \frac{1}{s - a} $ is the Laplace transform of $ e^{at} $:
    
    $$ L^{-1} \{M_{X} (-s)\}  = \frac{2}{3} e^{-2x} + 2 e^{-3x} = f_{X}(x) $$
    
    \item An exponential random variable with parameter $ \lambda $ has the PDF of $ \lambda e^{-\lambda x}, x \geq 0 $.
    
    Note for the 1st moment:
    
    $$ E[X] = \int_{0}^{\infty} x \lambda e^{-\lambda x} dx = \frac{1}{\lambda} $$
    
    Also for the 2nd moment, using integration by parts and reusing the result for the 1st moment:
    
    $$ E[X^2] = \int_{0}^{\infty} x^2 \lambda e^{-\lambda x} dx = -x^2 e^{-\lambda x}\Big|_{0}^{\infty} + \int_{0}^{\infty} 2x e^{-\lambda x} dx = 0 + \frac{2}{\lambda} \int_{0}^{\infty} x \lambda e^{-\lambda x} dx = \frac{2}{\lambda} \cdot \frac{1}{\lambda} = \frac{2}{\lambda^2} $$

    Find the 3rd moment, using integration by parts and reusing the result from the 2nd moment:
    
    $$ E[X^3] = \int_{0}^{\infty} x^3 \lambda e^{-\lambda x} dx = -x^3 e^{-\lambda x}\Big|_{0}^{\infty} + \int_{0}^{\infty} 3x^2 e^{-\lambda x} dx =  0 + \frac{3}{\lambda} \int_{0}^{\infty} x^2 \lambda e^{-\lambda x} dx = \frac{3}{\lambda} \cdot \frac{2}{\lambda^2} = \frac{6}{\lambda^3} $$
    
    Find the 4th moment, using integration by parts and reusing the result from the 3rd moment:
    
    $$ E[X^4] = \int_{0}^{\infty} x^4 \lambda e^{-\lambda x} dx  = -x^4 e^{-\lambda x}\Big|_{0}^{\infty} + \int_{0}^{\infty} 4 x^3 e^{-\lambda x} dx  =   0 + \frac{4}{\lambda} \int_{0}^{\infty} x^3 \lambda e^{-\lambda x} dx = \frac{4}{\lambda} \cdot \frac{6}{\lambda^3} = \frac{24}{\lambda^4} $$
    
    Find the 5th moment, using integration by parts and reusing the result from the 4th moment:
    
    $$ E[X^5] = \int_{0}^{\infty} x^5 \lambda e^{-\lambda x} dx  = -x^5 e^{-\lambda x}\Big|_{0}^{\infty} + \int_{0}^{\infty} 5 x^4 e^{-\lambda x} dx  =   0 + \frac{5}{\lambda} \int_{0}^{\infty} x^4 \lambda e^{-\lambda x} dx = \frac{5}{\lambda} \cdot \frac{24}{\lambda^3} = \frac{120}{\lambda^5} $$
    
    \item \begin{enumerate}
        \item The PDF of the standard normal is:
        
        $$ f_{X} (x) = \frac{1}{\sqrt{2\pi}} e^{-\frac{1}{2} x^2} $$
        
        Which integrates to 1:
        
        $$ \int_{-\infty}^{\infty} \frac{1}{\sqrt{2\pi}} e^{-\frac{1}{2} x^2} dx = 1 $$
        
        From $ Var[X] = E[X^2] - E[X]^2, E[X] = 0 $:
        
        \begin{equation*}
            \begin{aligned}
            E[X^2] &= 1 \\
            &= \int_{-\infty}^{\infty} x^2 \frac{1}{\sqrt{2\pi}} e^{-\frac{1}{2} x^2} dx
            \end{aligned}
        \end{equation*}
        
        To find $ E[X^3] $, using integration by parts and the result from $ E[X^2] $:
        
        $$ E[X^3] = \int_{-\infty}^{\infty} x^3 \frac{1}{\sqrt{2\pi}} e^{-\frac{1}{2} x^2} dx = -\frac{x^2}{\sqrt{2\pi}} e^{-\frac{1}{2} x^2} \Big|_{-\infty}^{\infty} + 2 \int_{-\infty}^{\infty} x \frac{1}{\sqrt{2\pi}} e^{-\frac{1}{2} x^2} dx = 0 $$
        
        To find $ E[X^4] $, using integration by parts and the result from $ E[X^3] $:
        
        $$ E[X^4] = \int_{-\infty}^{\infty} x^4 \frac{1}{\sqrt{2\pi}} e^{-\frac{1}{2} x^2} dx = -\frac{x^3}{\sqrt{2\pi}} e^{-\frac{1}{2} x^2} \Big|_{-\infty}^{\infty} + 3 \int_{-\infty}^{\infty} x^2 \frac{1}{\sqrt{2\pi}} e^{-\frac{1}{2} x^2} dx = 0 + 3 \cdot 1 = 3 $$
        
        \item Given $Y = a + b X + c X^2 $:
        
        \begin{equation*}
            \begin{aligned}
            E[Y] &= E[a + b X + c X^2] \\
            &= E[a] + E[b X] + E[c X^2] \\
            &= a + b E[X] + c E[X^2] \\
            &= a + 0 + c \\
            &= a + c
            \end{aligned}
        \end{equation*}
        
        To find $ \sigma_{Y} $:
        
        \begin{equation*}
            \begin{aligned}
            Var[Y] &= E[Y^2] - E[Y]^2 \\
            &= E[(a + b X + c X^2)^2] - (a + c) \\
            &= E[a^2 + a b X + a c X^2 + a b X + b^2 X^2 + b c X^3 + a c X^2 + b c X^3 + c^2 X^4] - (a + c) \\
            &= a^2 + 0 + ac + 0 + b^2 + 0 + a c + 0 + 3 c^2 - (a + c) \\
            &= a^2 + 2 a c + b^2 + 3 c^2 - a - c \\
            \sigma_{Y} &= \sqrt{Var[Y]} = \sqrt{a^2 + 2 a c + b^2 + 3 c^2 - a - c}
            \end{aligned}
        \end{equation*}
        
        Hence:
        
        \begin{equation*}
            \begin{aligned}
                \rho(X, Y) &= \frac{E[(X - \mu_{X})(Y - \mu_{Y})]}{\sigma_{X}\sigma_{Y}} \\
                &= \frac{E[X(a + b X + c X^2 - a - c)]}{\sqrt{a^2 + 2 a c + b^2 + 3 c^2 - a - c}} \\
                &= \frac{E[b X^2 + c X^3 - c X]}{\sqrt{a^2 + 2 a c + b^2 + 3 c^2 - a - c}} \\
                &= \frac{-c E[X] + b E[X^2] + c E[X^3]}{\sqrt{a^2 + 2 a c + b^2 + 3 c^2 - a - c}} \\
                &= \frac{b}{\sqrt{a^2 + 2 a c + b^2 + 3 c^2 - a - c}}
            \end{aligned}
        \end{equation*}
        
    \end{enumerate}
    
\end{enumerate}

\end{document}
